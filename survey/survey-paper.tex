\documentclass[11pt,conference]{IEEEtran}
\IEEEoverridecommandlockouts
\usepackage{cite}
\usepackage{amsmath,amssymb,amsfonts}
\usepackage{graphicx}
\usepackage{textcomp}
\usepackage{soul}
\usepackage{xcolor}
\def\BibTeX{{\rm B\kern-.05em{\sc i\kern-.025em b}\kern-.08em
T\kern-.1667em\lower.7ex\hbox{E}\kern-.125emX}}
\begin{document}

\title{Machine Learning Security Survey \\
}

\author{\IEEEauthorblockN{Turhan Kimbrough}
\IEEEauthorblockA{\textit{Department of Computer Science} \\
\textit{Towson University}\\
Towson, Maryland\\
tkimbr1@students.towson.edu}
}

\maketitle

\begin{abstract}
    Machine learning has experienced a significant growth in usage 
    over the past few decades. Due to its data-centric approach in modeling, 
    machine learning has seen use in a variety of subfields in computer science. In
    particular, researchers have been interested in incorporating machine learning
    into the domain of cybersecurity, utilizing it from the perspective of an
    adversary or ally. Researchers have also been concerned with the security state
    of current machine learning models. This survey paper provides an overview of
    machine learning, attacks and mitigations to current machine learning
    systems, machine learning use cases for adversaries, and future research
    directions.
\end{abstract}

\begin{IEEEkeywords}
    Neural networks, security, classification, machine learning model, training,
    validation
\end{IEEEkeywords}

\section{Introduction}
With the increasing widespread adoption of machine learning technology, its
usage is being observed in a variety of different fields. Some notable examples
include image recognition, voice assistant technologies, email spam filters, and search
engines. Much of its recent popularity can be attributed to the availability of
frameworks such as TensorFlow, a python-based library which provides
easy-to-use bindings for complex operations.

However, one growing concern tied to the ubiquity of machine learning is its 
accessibility to adversaries. Based on the assessment of current and prior
research, there are a number of vulnerabilities in current machine learning
models which can be exploited with little knowledge of a system's domain. In
addition, attackers have been able to leverage machine learning technology to
assist with the deployment of cyberattacks. Enterprise machine learning applications
may often contain large datasets of important information, becoming a potential 
candidate of a targeted attack. This paper will survey the domain of
cybersecurity with regard to machine learning. This topic will explore the
fundamentals of machine learning, current attacks on machine learning
systems, mitigations to machine learning attacks, machine learning technology from the perspective of adversaries, and
future research directions.

\section{Overview}
This section will introduce the fundamentals of machine learning. Additionally,
there will be a discussion about the key terms used by members of the community.

\subsection{Machine Learning Basics}
Traditionally, when software developers are tasked with solving a problem, they
use a combination of rules and logic to find a solution. The basic
routine consists of finding appropriate input values,
creating the logic and rules to process the input, and producing the appropriate output.
The traditional approach to software development allows for fine-tuned control
of program behavior to achieve the solution. However, this approach does not scale
with the complexity of additional rules and/or possible solutions. An
example is image classification, where the logic needed to compare images
is complex. This becomes a bigger concern when new classifications need to be
derived with new data. Machine learning flips the traditional programming
paradigm on its head, by taking a series of solutions as input, and
letting the machine develop the rules by detecting patterns in the solutions
[1].
The result is a self-propagating mathematical model, capable of making
decisions on newly supplied data. This is what is known as a \emph{machine
learning model}, a software component which can be saved to a file and
integrated into larger software systems [1].

Machine learning models can be implemented using a number of different
algorithms, some popular examples will be discussed in this paragraph. \emph{Regression algorithms} use statistics to model the
relationships between different variables plotted on a graph [2]. \emph{Decision
tree algorithms} represent the problems using trees, where internal nodes represent conditional statements and leaf nodes
represent decisions [3]. Lastly, \emph{artificial neural networks} are used in a
subset of machine learning known as \emph{deep learning}, where the algorithms
are modeled to mimic the structure of biological neural networks found in
animal brains [4].  

Briefly mentioned earlier, a common use case for machine learning is
\emph{classification}, where a dataset is categorized into different groups
based on one or more \emph{features}. A \emph{feature} is defined as some
measurable property or characteristic being observed in a dataset [1]. There are several types of
classifications, including \emph{binary classification}, \emph{multi-class
classification}, and \emph{multi-label classification}. \emph{Binary
classification} categorizes data based on whether a feature is present or not,
resulting in an outcome of true or false [1]. \emph{Multi-class classification}
categorizes data into different groups, where each data instance is assigned
according to a single feature [1]. \emph{Multi-label classification} categorizes data
into different groups, where each data instance is assigned according to its
expression of two or more features [1].
\emph{Learning} is the procedure for training algorithms to detect patterns in
data. 

There are three main types of training mechanisms, \emph{supervised
training}, \emph{unsupervised training}, and \emph{reinforcement learning}. \emph{Supervised training} requires
each data instance to have one or more labels, defining which category or
feature it expresses [5]. In contrast, \emph{unsupervised training} omits the need
for labels, and the machine learning model will categorize datasets on its own
[5].
\emph{Reinforcement learning} uses a reward-based system to maximize good
decisions when a software agent performs a task [5].
Typically, after the training phase, a machine learning model will go through
the process of \emph{validation}. \emph{Validation} typically consists of a separate dataset to guarantee the accuracy of a model and
preventing a phenomenon called \emph{over-fitting}, where a model will only
'memorize' characteristics or patterns of training data [5].

\section{Attacks to Machine Learning Models}
Due to the unique mechanism for constructing machine learning models, there are
several ways which they can be exploited by an attacker. As a result, the
attack surface which exists on machine learning models will differ from
traditional software systems.
This section will discuss several attack types which target machine learning
systems.

\subsection{Data Poisoning}
Data Poisoning is the act of manipulating, removing, or adding
data during the training phase of machine learning [6]. This type of attack is
known as a \emph{black box attack}, where an attacker does not need to know the
implementation of a system to attack it.

One popular instance of a data poisoning attack is the adversarial
example. This requires the attacker to have some knowledge of the
training data, such as its dimensions and data type. The attacker would then
manipulate data instances in a way where the machine learning model would
be fooled, but appears normal to a human observer [6]. The reason for manipulating
data instances in this way is two-fold. First, machine learning models are
often constrained to data fitting a specific dimension, shape, size, or length
of characters. This is typically done to prevent incompatible data from
entering the model during training. Second, there is often one or more people
who are observing the model with testing or validation datasets. An attacker
would want to minimize any evidence of the data being tampered with.

Often, the goal of this
attack is to make a machine learning model incorrectly classify data. The
consequences of this attack can be devastating. 

Researchers at [17] use adversarial examples to trick a real-world image
classification model. The researchers used variations of image perturbations, where noise
and/or pixel values are slightly modified to affect the detected features in an
image. As a result, images were misclassified even when the disturbances were
applied to images originating from a smartphone camera.

\subsection{Membership Inference}
Membership inference is a mechanism of data extraction, where an attacker 
intends to know whether certain samples were used as training data for a
machine learning model [7]. This type of attack is classified as a \emph{gray
box attack}, where the attacker has some knowledge of the software system.

Researchers at [18] evaluated publicly available models to detect the
presence of data characteristics in hypothetical datasets. The researchers
built several 'shadow' models using parameters that closely resembled the target
machine learning model. The shadow models would take the output prediction of
the target model as inputs, and distinguish the characteristics between how the
target evaluated training data and new data. As a result, the shadow models
were able to detect when the target exhibited over-fitting, indicating a data
member was part of the training set.

Often, this can exploit the confidentiality of information on a system. An
attacker has the capability to correlate information between datasets to target
individuals for other cyberattacks.


\subsection{Transfer Learning}
Transfer learning is a mechanism where an attacker has the ability to study a
publicly available machine learning model, and use that insight to sneak past
and/or corrupt similar target systems [8]. This type of attack is classified as a
\emph{white box attack}, since the attacker would need to have full access to
at least one machine learning model.

This particular attack requires knowledge of the
input data, learning mechanism, and output behavior of a machine learning model
similar to the target. Once this information is obtained, the attacker would
test the system and learn its overall behavior to enumerate flaws in the
model's logic [8]. The assumption is that the flaws found in the available model's
logic would appear in a target system.

Often, the goal of this attack is similar to the other two mentioned above. An
attacker would use the information to trick, fool, manipulate, or sneak passed
a target machine learning model.

Researchers at [19] use the transfer learning technique to launch backdoor
attacks on a target machine learning model. The researchers used insight from
the access they had to a
neural network structure closely resembling the target. The attack
exploited the presence of extra neurons in the target model, which affect the
classification of the model when activated. Using carefully constructed input
data, a backdoor attack was launched causing a misclassification of images and
time series data.

\section{Mitigations to Attacks}

Due to the large attack surface found in machine learning models, there has
been a significant focus in its security. This section will
discuss several mitigations to the attacks discussed in the previous
section. 

This section will also reference a generic process known as the \emph{machine
learning lifecycle}. This lifecycle consists of building the dataset, feeding
the input to the model, training the model, verifying the model accuracy,
deployment, and maintenance.

\subsection{Sanitizing Input}
One mechanism for securing a machine learning system is to 
filter malicious samples from clean samples during the training phase. There
are several different approaches to sanitizing depending on the type of data
being fed to the model. A general procedure for sanitizing input data is still
an open research topic.

One naive approach for filtering data is the use of \emph{gradient masking}.
This simple technique will manipulate each input sample to create a sharper
decision boundary for the machine learning model to work on. A common
implementation of this technique for images is \emph{binary thresholding},
where each pixel in an image is converted to black or white depending on its
color value. This technique will mitigate against perturbations to data
[9].

Another approach called ANTIDOTE, proposed by [11] uses an \emph{anomaly-based}
%Rubinstein \emph{et al} uses an \emph{anomaly-based}
detection scheme to characterize data samples. ANTIDOTE uses statistics to
differentiate between normal and malicious samples. Once a malicious sample is
detected, it is automatically discarded from the model.

\subsection{Machine Learning Retraining}
After a machine learning model is finished with its initial training and
verification, it is typically ready to be deployed. After deployment, a good
practice to maintain a machine learning model is to periodically retrain it.
Retraining can either use the original dataset, or a new dataset containing
similar information as the original. This is ideal for machine learning models
which continuously learn post-deployment.

A specialized approach to retraining using samples of malicious data is a technique called
\emph{adversarial retraining}. This process requires  
a sufficient number of malicious samples with perturbations, along with the
original training data. The technique
involves labelling malicious data with perturbations as adversarial samples, and training
the model to detect them appropriately [10]. This ensures that adversarial samples will not affect
the overall accuracy of the model, by learning to filter perturbations added to
any new data [10]. The model will learn to differentiate between legitimate and
adversarial data when re-deployed.

Another area of interest for model retraining is the use of automation. In
particular, researchers at [20] demonstrate the ability to re-train a
machine learning model without knowledge of its inner-workings. In particular,
the \emph{Tree-based Pipeline Optimization Tool (TPOT)} is the core re-training
tool, which optimizes a machine learning model by slightly reconfiguring its
structure with new data. The researchers were able to continuously optimize an
advertisement component utilizing machine learning in the Spotify music
streaming service.

\subsection{Differential Privacy}
During deployment, data will typically be flowing in and out of a machine
learning model. Since information associated with a machine learning system may be
sensitive, systems will typically incorporate an access control mechanism to interface with it. Even if a system
is robust, a data leak will completely compromise confidentiality. To
circumvent this issue, \emph{differential privacy} is a mechanism to
effectively store
information from a dataset while hiding confidential details [12].

The implementation of differential privacy will differ depending on the data
being processed. One mechanism for differential privacy is to present
output data showing non-sensitive information normally, while sensitive
information is encrypted. Another mechanism is presenting the output
information of a machine learning model as a reference to another dataset, which
will then need to be queried with subsequent processing [21].
\hl{ADD RESEARCH HERE}

\section{Assisting Cyberattacks with Machine Learning}
After discussing the attacks/mitigations for machine learning systems
from the perspective of the defender, this section will discuss how machine
learning technology is used to assist with the deployment of cyberattacks from
the perspective of the attacker.

\subsection{Evasive Malware}
Modern-day malware detection schemes use several different techniques to
classify malicious and safe programs. One method is the use of detection
engines, which are often available in two distinct types,
\emph{signature-based} detection engines and \emph{anomaly-based} detection
engines. A signature-based detection engine keeps a record of malicious
behaviors and footprints, comparing software behaviors against the record to
identify malware [13]. On the other hand, an anomaly-based engine will
compare software behaviors against a record of 'normal' system behavior to
detect deviations, classifying them as malicious actions [13]. Both approaches can be extended with machine learning
capabilities to improve their detection rates. In particular, supervised
learning has been shown to be an effective strategy.

However, even with the advancements in malware detection schemes, research
has shown that there can be a significant flaw associated with detection
engines using static features and definitive labels. In particular, researchers
at [13] created \emph{DQEAF}, a framework which has demonstrated the ability to evade malware-detection by
the use of reinforcement learning. DQEAF operates by using a machine
learning agent to interact with different malware samples. During its
interaction, it will slightly modify the behaviors of detected malware samples
without impacting its overall structure and/or functionality. The agent will
continuously modify detected malware samples until it is able to be completely
undetected by a malware detection engine.

\subsection{CAPTCHA Bypass}
CAPTCHA is an acronym which stands for the "Completely Automated Public Turing test
to tell Computers and Humans Apart". CAPTCHAs are a 
mechanism to combat against the growing number bots which are used on the
internet, ideally providing a challenge which is relatively easy for humans to
solve, but difficult for machines. CAPTCHAs come in a variety of types,
including the deciphering of obfuscated text, interpreting audio messages,
selecting images based on descriptions, and the tracking of end-user behavior.

Over the past few years, CAPTCHAs have grown to be more difficult due to the
techniques used by adversaries. The techniques include the exploitation of
weaknesses in the CAPTCHA generating mechanism
and using machine learning for CAPTCHA solving. Machine learning
CAPTCHA solving is an especially attractive option, given the automation
capabilities. In fact, researchers at [14] suggested that reinforcement learning has
provided the capability to solve one of Google's reCAPTCHA mechanisms, which
records mouse movements to distinguish human and bot behavior. This was
achievable by representing the pixels on a screen as a matrix, and using a
software agent to move the
cursor across the screen. The \emph{Markov
Decision Process} is the algorithm which was used to generate random movements,
and trains the software agent to behave more human-like over time. This
process solved reCAPTCHAs with greater than 90\% accuracy.

\subsection{Brute Forcing Passwords}
One of the most prevalent cybercrimes in today's age is account hijacking. From
banking, social media, streaming, and more, digital accounts play a major role
in the lives of everyday people. Adversaries are interested in
gaining access to these accounts in an effort to unravel useful information.
With the works of prior research and the open source community, there are a
significant number of tools available for account hijacking. One of the
simplest tools are \emph{brute force password-crackers}, which typically employ a random
combination of characters to guess the resulting password.

Due to the increasing awareness in cybersecurity best practices, many sites
with accounts will employ mechanisms to enforce stronger passwords. As a
result, simple password crackers are no longer able to make guesses 
within a reasonable amount of time. However, researchers at [15] suggest that a
generative machine learning model is capable of creating more accurate guesses
based on previously exposed password datasets. The approach uses the
\emph{Markov Model} with a neural network to derive characteristics from a
password dataset, and generate new passwords with the same characteristics.
Comparing the generated passwords against a separate leaked dataset, the
researchers were able to match up to 42\% of the passwords using the generative
machine learning model.

\section{Future Work}
Much of the research which has been conducted in the field of machine learning
security has been relatively new. As a result, the research community still has
many questions as to what components are necessary and/or feasible for securing
future machine learning systems.
This section will outline potential directions for future research based on
what has been reviewed in prior sections.

\subsection{Trusted Platforms}
Currently, the usage of \emph{Trusted Platform Modules} (TPMs) are seeing a
rise in popularity for securing systems. TPM technology typically consists of a
set of cryptographic algorithms along with a dedicated processor known as a TPM
chip. The purpose of a TPM chip is to provide a dedicated hardware device to
independently verify that a software system has not been tampered with. 

An interesting research direction would be to incorporate TPMs into a machine
learning system. In addition, there has been a recent surge in mobile devices
which ship with dedicated machine learning processing units, such as flagship
Android smartphones which use machine learning for photography/videography.
It would not be surprising if researchers would be interested in using a TPM
and machine learning processing unit as building blocks for a trusted platform.

\subsection{GAN-assisted Model Retraining}
The concept of Generative Adversarial Networks (GANs) is still a relatively
new, being discovered only within the last decade. Generative Adversarial
Networks consist of two separate neural networks, a \emph{generator}
and \emph{discriminator}. Given an initial training data set, the goal is to
produce new data which mimic the characteristics of the training set. This is
accomplished by using the \emph{generator} to produce realistic fake data from
a random seed, and the \emph{discriminator} which learns to differentiate the
fake data and real data [16]. The two networks are in an adversarial relationship,
hence the name, and work toward minimizing the differences between the fake and
real data [16]. This concept has brought significant insight into the inner-workings
of machine learning, along with insights on the relationships between data
samples.

An interesting research direction would be to use Generative Adversarial
Networks to aid in model retraining. In order to do this, a
variety of malicious data samples would need to first be aggregated. Then the
\emph{generator} and \emph{discriminator} pair would work to create a model
which can realistically create new malicious data samples. This new (malicious)
model can then be used to generate labeled malicious data, which will be fed to
machine learning model which needs to undergo retraining. The machine learning
model which is being retrained will have a more reliable malicious dataset,
which can strengthen its detection mechanism.

\subsection{Reinforcement Learning Framework for Cyber Defense}
One of the biggest challenges in cybersecurity is finding a general software
solution which can defend against new threats. Especially in the
context of \emph{zero-day exploits}, which are cyberattacks that occur before a
software vulnerability is well-known to the public. Additionally, even known
malware samples may pass through a system undetected due to the nature of
anti-malware systems.

An interesting research direction would be to incorporate a general framework
for securing a software system through the use of reinforcement learning. This
type of learning would be ideal in a live software environment due to the
near-infinite number of possibilities for which it can be attacked. Specifically, it
would act more as an aid to patching software vulnerabilities, suggesting where
they could occur based on internet data and assessments.

\section{Conclusion}
In this paper, the topic of machine learning security from different
perspectives was explored. From the perspective of someone new to machine
learning, an overview was given. From the perspective of a defender, the
vulnerabilities and mitigations for machine learning systems was given. The
perspective of adversarial use-cases for machine learning was also portrayed. Lastly,
prospective work for researchers have been included as well. It's clear that
machine learning is the way forward for many software solutions, and it is
important to consider how its ubiquity will affect the domain of cybersecurity. 


\begin{thebibliography}{00}
    \bibitem{b1} E. Alpaydin, Introduction to machine learning, MIT press,
        2014.
    \bibitem{b2} E. Gambhir, R. Jain, A. Gupta and U. Tomer, "Regression
        Analysis of COVID-19 using Machine Learning Algorithms," 2020
        International Conference on Smart Electronics and Communication
        (ICOSEC), 2020, pp. 65-71, doi: 10.1109/ICOSEC49089.2020.9215356.
    \bibitem{b3} R. C. Barros, M. P. Basgalupp, A. A. Freitas and A. C. P. L.
        F. de Carvalho, "Evolutionary Design of Decision-Tree Algorithms
        Tailored to Microarray Gene Expression Data Sets," in IEEE Transactions
        on Evolutionary Computation, vol. 18, no. 6, pp. 873-892, Dec. 2014,
        doi: 10.1109/TEVC.2013.2291813.
    \bibitem{b4} H. B. Demuth, M. H. Beale, O. De Jess, and M. T.
        Hagan, Neural network design, Martin Hagan, 2014.
    \bibitem{b5} W. Grant and W. Yu, “A Survey of Deep Learning: Platforms,
        Applications and Emerging Research Trends; A Survey of Deep Learning:
        Platforms, Applications and Emerging Research Trends,” 2018, doi:
        10.1109/ACCESS.2018.2830661.
    \bibitem{b6} C. Liu, B. Li, Y. Vorobeychik, and A. Oprea, "Robust linear
        regression against  training  data  poisoning,"  in Proceedings  of
        the  10th  ACM Workshop  on  Artificial  Intelligence  and  Security,
        2017,  pp.  91-102: ACM.
    \bibitem{b7} R. Shokri, M. Stronati, C. Song and V. Shmatikov, "Membership
        Inference Attacks Against Machine Learning Models," 2017 IEEE Symposium
        on Security and Privacy (SP), 2017, pp. 3-18, doi: 10.1109/SP.2017.41.
    \bibitem{b8} B. Wu, A. Shuo Wang, A. Xingliang Yuan, C. Wang, C.
        Rudolph, and A. Xiangwen Yang, “Towards Defeating Misclassification
        Attacks Against Transfer Learning.”
    \bibitem{b9} Y. Yanagita, M. Yamamura, "Gradient
        Masking Is a Type of Overfitting," 2018 International Journal of Machine
        Learning and Computing. 8. 203-207. 10.18178/ijmlc.2018.8.3.688.
    \bibitem{b10} S. H. Silva and P. Najafirad, “Opportunities and
        Challenges in Deep Learning Adversarial Robustness: A Survey.”
    \bibitem{b11} B. Rubinstein, B. Nelson , L. Huang, J. Anthony, L.
        Shing-hon, N. Taft, J. Tygar, "ANTIDOTE:
        understanding and defending against poisoning of anomaly detectors,"
        2009, Proceedings of the ACM SIGCOMM Internet Measurement Conference, IMC.
        1-14. 10.1145/1644893.1644895. 
    \bibitem{b12} M. Abadi et al., “Deep Learning with Differential Privacy,”
        2016, doi: 10.1145/2976749.2978318.
    \bibitem{b13} Z. Fang, J. Wang, B. Li, S. Wu, Y. Zhou and H. Huang,
        "Evading Anti-Malware Engines With Deep Reinforcement Learning," in
        IEEE Access, vol. 7, pp. 48867-48879, 2019, doi:
        10.1109/ACCESS.2019.2908033.
    \bibitem{b14} I. Akrout, A. Feriani, and M. Akrout, “Hacking  Google
        re-CAPTCHA v3 using Reinforcement Learning.”
    \bibitem{b15} B. Hitaj, P. Gasti, G. Ateniese, F. Perez-Cruz, "PassGAN: A Deep Learning Approach for
        Password Guessing," 2017
    \bibitem{b16} I. Goodfellow, J. Pouget-Abadie, M. Mirza, B. Xu, D.
        Warde-Farley, S. Ozair, A. Courville, Y. Bengio, "Generative
        Adversarial Networks," 2014, Advances in Neural
        Information Processing Systems. 3. 10.1145/3422622. 
    \bibitem{b17} A. Kurakin, I. Goodfellow, S. Bengio, "Adversarial examples
        in the physical world," 2016
    \bibitem{b18} R. Shokri, M. Stronati, C. Song and V. Shmatikov, "Membership
        Inference Attacks Against Machine Learning Models," 2017 IEEE Symposium
        on Security and Privacy (SP), 2017, pp. 3-18, doi: 10.1109/SP.2017.41.
    \bibitem{b19} S. Wang, S. Nepal, C. Rudolph, M. Grobler, S. Chen and T.
        Chen, "Backdoor Attacks against Transfer Learning with Pre-trained Deep
        Learning Models," in IEEE Transactions on Services Computing, doi:
        10.1109/TSC.2020.3000900.
    \bibitem{b20} A. Kavikondala, V. Muppalla, K. Prakasha, V. Acharya, Automated Retraining of Machine
        Learning Models," 2019, 8. 445-452. 10.35940/ijitee.L3322.1081219. 
    \bibitem{b21} M. Gong, Y. Xie, K. Pan, K. Feng and A. K. Qin, "A Survey on
        Differentially Private Machine Learning [Review Article]," in IEEE
        Computational Intelligence Magazine, vol. 15, no. 2, pp. 49-64, May
        2020, doi: 10.1109/MCI.2020.2976185.

\end{thebibliography}
\vspace{12pt}
\end{document}
