\documentclass[11pt,conference]{IEEEtran}
\IEEEoverridecommandlockouts
% The preceding line is only needed to identify funding in the first footnote. If that is unneeded, please comment it out.
\usepackage{cite}
\usepackage{amsmath,amssymb,amsfonts}
\usepackage{graphicx}
\usepackage{textcomp}
\usepackage{xcolor}
\def\BibTeX{{\rm B\kern-.05em{\sc i\kern-.025em b}\kern-.08em
    T\kern-.1667em\lower.7ex\hbox{E}\kern-.125emX}}
\begin{document}

\title{Machine Learning Security Survey \\
}

\author{\IEEEauthorblockN{Turhan Kimbrough}
\IEEEauthorblockA{\textit{Department of Computer Science} \\
\textit{Towson University}\\
Towson, Maryland\\
tkimbr1@students.towson.edu}
}

\maketitle

\begin{abstract}
Machine learning has experienced a significant growth in usage 
over the past few decades. Due to its data-centric approach in modeling, 
machine learning has seen use in a variety of subfields in computer science. In
particular, researchers have been interested in incorporating machine learning
into the domain of cybersecurity, utilizing it from the perspective of an
adversary or ally. Researchers have also been concerned with the security state
of current machine learning models. This survey paper provides an overview of
machine learning, vulnerabilities and mitigations to current machine learning
    systems, machine learning use cases for adversaries, and future research
    directions.
\end{abstract}

\begin{IEEEkeywords}
Neural networks, security, classification, machine learning model, training,
    validation
\end{IEEEkeywords}

\section{Introduction}
With the increasing widespread adoption of machine learning technology, its
usage is being observed in a variety of different fields. Some notable examples
include image recognition, voice assistant technologies, email spam filters, and search
engines. Much of its recent popularity can be attributed to the availability of
frameworks such as tensorflow, allowing people of almost any background to
quickly draft a machine learning application.

However, one growing concern tied to the ubiquity of machine learning is its 
accessibility to adversaries. Based on the assessment of current and prior
research, there are a number of vulnerabilities in current machine learning
models which can be exploited with little knowledge of a system's domain. In
addition, attackers have been able to leverage machine learning technology to
assist the deployment of cyberattacks. Enterprise machine learning applications
may often contain large datasets of important information, becoming a potential 
candidate of a targeted attack. This paper will survey the domain of
cybersecurity with regard to machine learning. This topic will explore the
fundamentals of machine learning, current vulnerabilities in machine learning
systems, mitigations to machine learning vulnerabilities, machine learning technology from the perspective of adversaries, and
future research directions.

\section{Overview}
This section will introduce the fundamentals of machine learning. Additionally,
there will be a discussion about the key terms used by members of the community.

\subsection{Machine Learning Basics}
Traditionally, when software developers are tasked with solving a problem, they
use a combination of rules and logic to find a solution. The basic
routine consists of finding appropriate input values,
creating the logic and rules to process the input, and producing the appropriate output.
The traditional approach to software development allows for fine-tuned control
of program behavior to achieve the solution. However, this approach does not scale
with the complexity of additional rules and/or possible solutions. An
example is image classification, where the logic needed to compare images
is complex. This becomes a bigger concern when new classifications need to be
derived with new image data. Machine learning flips the traditional programming
paradigm on its head, by taking a series of solutions as input, and
letting the machine develop the rules by detecting patterns in the solutions.
The result is a self-propagating mathematical model, capable of making
decisions on newly supplied data. The mathematical model is often represented
as an \emph{Artificial Neural Network} (ANN), whose name is inspired by the
biological brain, mimicking the way neurons interact with one another. ANN's comprise of an input layer, one
or more hidden layers for data processing, and an output layer for
decision-making. This approach relies on large sets of
well-defined data, and has the flexibility for being used in many different applications.

Briefly mentioned earlier, a common use case for machine learning is
\emph{classification}, where a dataset is categorized into different groups
based on one or more \emph{features}. A \emph{feature} is defined as some
measurable property or characteristic being observed in a dataset. There are several types of
classifications, including \emph{binary classification}, \emph{multi-class
classification}, and \emph{multi-label classification}. \emph{Binary
classification} categorizes data based on whether a feature is present or not,
resulting in an outcome of true or false. \emph{Multi-class classification}
categorizes data into different groups, where each data instance is assigned
according to its feature. \emph{Multi-label classification} categorizes data
into different groups, where each data instance is assigned according to its
expression of on or more features.

\emph{Training} is the process of teaching a machine learning model to detect patterns
in datasets. There are two types of training mechanisms, \emph{supervised
training} and \emph{unsupervised training}. \emph{Supervised training} requires
each data instance to have one or more labels, defining which category or
feature it expresses. In contrast, \emph{unsupervised training} omits the need
for labels, and the machine learning model will categorize datasets on its own.
Typically, after the training phase, a machine learning model will go through
the process of \emph{validation}. \emph{Validation} typically consists of
classifying a separate dataset to guarantee the accuracy of a model and
preventing a phenomenon called \emph{over-fitting}, where a model will only
'memorize' characteristics or patterns of training data.

\section{Vulnerabilities in Machine Learning Models}
Due to the unique mechanism for constructing machine learning models, there are
several ways which they can be exploited by an attacker. As a result, the
attack surface which exists on machine learning models will differ from
traditional software systems.
This section will discuss several vulnerabilities which are currently present
in machine learning.

\subsection{Data Poisoning}
Data Poisoning is the act of manipulating, removing, or adding
data during the training phase of machine learning. This type of attack is
known as a \emph{black box attack}, where an attacker does not need to know the
implementation of a system to attack it.

One popular instance of a data poisoning attack is the adversarial
example. This requires the attacker to have some knowledge of the
training data, such as its dimensions and data type. The attacker would then
manipulate data instances in a way where the machine learning model would
be fooled, but appears normal to a human observer. The reason for manipulating
data instances in this way is two-fold. First, machine learning models are
often constrained to data fitting a specific dimension, shape, size, or length
of characters. This is typically done to prevent incompatible data from
entering the model during training. Second, there is often one or more people
who are observing the model with testing or validation datasets. An attacker
would want to minimize any evidence of the data being tampered with.

Often, the goal of this
attack is to make a machine learning model incorrectly classify data. The
consequences of this attack can be devastating. Examples include tricking an
autonomous vehicle to misinterpret traffic signs, sneaking malicious data past an
intrusion detection system, or bypassing email spam filters.

\subsection{Membership Inference}
Membership inference is a mechanism of data extraction, where an adversary
intends to know whether certain samples were used as training data for a
machine learning model. This type of attack is also classified as a \emph{black box attack}.

This particular vulnerability requires significant effort from the attacker.
The attacker would need access to a dataset of sufficient size mimicking the
data in the target model. The data would then be used to create several
\emph{shadow models}, which are used only to recognize differences
in the target model's behavior. This is done to expose \emph{overfitting}, when a model's analysis corresponds too
closely to its training data. Therefore, an attacker can interpret whether
certain samples were used in the training dataset based on the target model's
confidence level in classifications.

Often, this can exploit the confidentiality of information on a system. An
attacker has the capability to correlate information between datasets to target
individuals for other cyberattacks.

\subsection{Transfer Learning}
Transfer learning is a mechanism where an adversary has the ability to study a
publicly available machine learning model, and use that insight to sneak past
and/or corrupt similar target systems. This type of attack is classified as a
\emph{white box attack}, since the attacker would need to have full access to
at least one machine learning model.

This particular vulnerability requires the attacker to have knowledge of the
input data, learning mechanism, and output behavior of a machine learning model
similar to the target. Once this information is obtained, the attacker would
test the system and learn its overall behavior to enumerate flaws in the
model's logic. The assumption is that the flaws found in the available model's
logic would appear in a target system.

Often, the goal of this attack is similar to the other two mentioned above. An
attacker would use the information to trick, fool, manipulate, or sneak passed
a target machine learning model.

\section{Mitigations to Vulnerabilities}

Due to the large attack surface found in machine learning models, there has
been a significant focus in its security. This section will
discuss several mitigations to the vulnerabilities discussed in the previous
section. 

This section will also reference a generic process known as the \emph{machine
learning lifecycle}. This lifecycle consists of building the dataset, feeding
the input to the model, training the model, verifying the model accuracy,
deployment, and maintenance.

\subsection{Selecting Trusted Datasets}
To secure a machine learning system, it is important to start at the beginning
of the machine learning lifecycle, securing the input data itself. Since
the training phase typically requires a large amount of data,
many projects utilize publicly available datasets. Publicly available datasets
are similar to open source projects, in that contributions are made by and
verified by a community of individuals. This means that the trust is placed on
the community to provide legitimate information. To find trusted datasets, one
can check the policy of verifying contributions, number of community
members/maintainers, and entities backing the project. For example, datasets
from the Government, Universities, and repositories such as kaggle are
typically trustworthy.

If a dataset is chosen from a location with an unknown reputation, it is
recommended to perform an audit to ensure the data consists of relevant
samples, proper labels, and a balanced set of classes.

\subsection{Sanitizing Input}
The next step in securing a machine learning system is to use mechanisms which
filter malicious samples from clean samples during the training phase. There
are several different approaches to sanitizing depending on the type of data
being fed to the model. A general procedure for sanitizing input data is still
an open research topic.

One naive approach for filtering data is the use of \emph{gradient masking}.
This simple technique will manipulate each input sample to create a sharper
decision boundary for the machine learning model to work on. A common
implementation of this technique for images is \emph{binary thresholding},
where each pixel in an image is converted to black or white depending on its
color value. This technique will mitigate against perturbations to data.

Another approach called ANTIDOTE, proposed by Rubinstein \emph{et al}, uses an \emph{anomaly-based}
detection scheme to characterize data samples. ANTIDOTE uses statistics to
differentiate between normal and malicious samples. Once a malicious sample is
detected, it is automatically discarded from the model.

\subsection{Machine Learning Retraining}
After a machine learning model is finished with its initial training and
verification, it is typically ready to be deployed. After deployment, a good
practice to maintain a machine learning model is to periodically retrain it.
Retraining can either use the original dataset, or a new dataset containing
similar information as the original. This is ideal for machine learning models
which continuously learn post-deployment.

A specialized approach to retraining using samples of malicious data is a technique called
\emph{adversarial retraining}. This process requires  
a sufficient number of malicious samples with perturbations, along with the
original training data. The technique
involves labelling malicious data with perturbations as adversarial samples, and training
the model to detect them appropriately. This ensures that adversarial samples will not affect
the overall accuracy of the model, by learning to filter perturbations added to
any new data. The model will learn to differentiate between legitimate and
adversarial data when re-deployed.

\subsection{Differential Privacy}
During deployment, data will typically be flowing in and out of a machine
learning model. Information coming in and out of a machine learning model may be sensitive, so
systems will typically incorporate APIs to interface with it. Even if a system
is robust, a data leak will completely compromise confidentiality. To
circumvent this issue, \emph{differential privacy} is a mechanism to
effectively store
information while hiding confidential details.

The implementation of differential privacy will differ depending on the data
being processed. A standard mechanism for differential privacy is to present
output data showing non-sensitive information normally, while sensitive
information is encrypted. Another mechanism is presenting the output
information of a machine learning model as a reference to another dataset, which
will then need to be queried with subsequent processing.

\section{Assisting Cyberattacks with Machine Learning}
After discussing the vulnerabilities/mitigations for machine learning systems
from the perspective of the defender, this section will discuss how machine
learning technology is used to assist with the deployment of cyberattacks from
the perspective of the attacker.

\subsection{Evasive Malware}
Modern-day malware detection schemes use several different techniques to
classify malicious and safe programs. One method is the use of detection
engines, which are often available in two distinct types,
\emph{signature-based} detection engines and \emph{anomaly-based} detection
engines. A signature-based detection engine keeps a record of malicious
behaviors and footprints, comparing software behaviors against the record to
identify malware. On the other hand, an anomaly-based engine will
compare software behaviors against a record of 'normal' system behavior to
detect deviations, classifying them as malicious actions. Both approaches can be extended with machine learning
capabilities to improve their detection rates. In particular, supervised
learning has been shown to be an effective strategy.

However, even with the advancements in malware detection schemes, research
has shown that there can be a significant flaw associated with detection
engines using static features and definitive labels. In particular, a framework
named \emph{DQEAF} has demonstrated the ability to evade malware-detection by
the use of reinforcement learning. DQEAF operates by using a machine
learning agent to interact with different malware samples. During its
interaction, it will slightly modify the behaviors of detected malware samples
without impacting its overall structure and/or functionality. The agent will
continuously modify detected malware samples until it is able to be completely
undetected by a malware detection engine.

\subsection{CAPTCHA Bypass}
CAPTCHA is an acronym which stands for the "Completely Automated Public Turing test
to tell Computers and Humans Apart". CAPTCHAs are a 
mechanism to combat against the growing number bots which are used on the
internet, ideally providing a challenge which is relatively easy for humans to
solve, but difficult for machines. CAPTCHAs come in a variety of types,
including the deciphering of obfuscated text, interpreting audio messages,
selecting images based on descriptions, and the tracking of end-user behavior.

Over the past few years, CAPTCHAs have grown to be more difficult due to the
techniques used by adversaries. The techniques include the exploitation of
weaknesses in the CAPTCHA generating mechanism
and using machine learning for CAPTCHA solving. Machine learning
CAPTCHA solving is an especially attractive option, given the automation
capabilities. In fact, research has suggested that reinforcement learning has
provided the capability to solve one of Google's reCAPTCHA mechanisms, which
records mouse movements to distinguish human and bot behavior. This was
achievable by representing the pixels on a screen as a matrix, and using a
software agent to move the
cursor across the screen. The \emph{Markov
Decision Process} is the algorithm which was used to generate random movements,
and trains the software agent to behave more human-like over time. This
process solved reCAPTCHAs with greater than 90\% accuracy.

\subsection{Brute Forcing Passwords}
One of the most prevalent cybercrimes in today's age is account hijacking. From
banking, social media, streaming, and more, digital accounts play a major role
in the lives of everyday people. Adversaries are interested in
gaining access to these accounts in an effort to unravel useful information.
With the works of prior research and the open source community, there are a
significant number of tools available for account hijacking. One of the
simplest tools are \emph{brute force password-crackers}, which typically employ a random
combination of characters to guess the resulting password.

Due to the increasing awareness in cybersecurity best practices, many sites
with accounts will employ mechanisms to enforce stronger passwords. As a
result, simple password crackers are no longer able to retrieve passwords
within a reasonable amount of time. However, research has suggested that by
using a simple supervised training mechanism, a password cracker can become
much more effective.

\section{Future Work}
Much of the research which has been conducted in the field of machine learning
security has been relatively new. As a result, the research community still has
many questions as to what components are necessary and/or feasible for securing
future machine learning systems.
This section will outline potential directions for future research based on
what has been reviewed in prior sections.

\subsection{Trusted Platforms}
Currently, the usage of \emph{Trusted Platform Modules} (TPMs) are seeing a
rise in popularity for securing systems. TPM technology typically consists of a
set of cryptographic algorithms along with a dedicated processor known as a TPM
chip. The purpose of a TPM chip is to provide a dedicated hardware device to
independently verify that a software system has not been tampered with. 

An interesting research direction would be to incorporate TPMs into a machine
learning system. In addition, there has been a recent surge in mobile devices
which ship with dedicated machine learning processing units, such as Apple
MacBooks powered by M1 processors and Samsung Galaxy S-series flagship devices.
It would not be surprising if researchers would be interested in using a TPM
and machine learning processing unit as building blocks for a trusted platform.

\subsection{Generative Adversarial Networks}
The concept of Generative Adversarial Networks is still a relatively
new, being discovered only within the last decade. Generative Adversarial
Networks consist of two separate neural networks, a \emph{generator}
and \emph{discriminator}. Given an initial training data set, the goal is to
produce new data which mimic the characteristics of the training set. This is
accomplished by using the \emph{generator} to produce realistic fake data from
a random seed, and the \emph{discriminator} which learns to differentiate the
fake data and real data. The two networks are in an adversarial relationship,
hence the name, and work toward minimizing the differences between the fake and
real data. This concept has brought significant insight into the inner-workings
of machine learning along with insights on the relationships between data
samples.

An interesting research direction would be to use Generative Adversarial
Networks to aid in model retraining. In order to do this, a
variety of malicious data samples would need to first be aggregated. Then the
\emph{generator} and \emph{discriminator} pair would work to create a model
which can realistically create new malicious data samples. This new (malicious)
model can then be used to generate labeled malicious data, which will be fed to
machine learning model which needs to undergo retraining. The machine learning
model which is being retrained will have a more reliable malicious dataset,
which can strengthen its detection mechanism.

\subsection{Reinforcement Learning Framework for Anti-malware Engines}

\section{Conclusion}


\begin{thebibliography}{00}
\bibitem{b1} G. Eason, B. Noble, and I. N. Sneddon, ``On certain integrals of Lipschitz-Hankel type involving products of Bessel functions,'' Phil. Trans. Roy. Soc. London, vol. A247, pp. 529--551, April 1955.
\bibitem{b2} J. Clerk Maxwell, A Treatise on Electricity and Magnetism, 3rd ed., vol. 2. Oxford: Clarendon, 1892, pp.68--73.
\bibitem{b3} I. S. Jacobs and C. P. Bean, ``Fine particles, thin films and exchange anisotropy,'' in Magnetism, vol. III, G. T. Rado and H. Suhl, Eds. New York: Academic, 1963, pp. 271--350.
\bibitem{b4} K. Elissa, ``Title of paper if known,'' unpublished.
\bibitem{b5} R. Nicole, ``Title of paper with only first word capitalized,'' J. Name Stand. Abbrev., in press.
\bibitem{b6} Y. Yorozu, M. Hirano, K. Oka, and Y. Tagawa, ``Electron spectroscopy studies on magneto-optical media and plastic substrate interface,'' IEEE Transl. J. Magn. Japan, vol. 2, pp. 740--741, August 1987 [Digests 9th Annual Conf. Magnetics Japan, p. 301, 1982].
\bibitem{b7} M. Young, The Technical Writer's Handbook. Mill Valley, CA: University Science, 1989.
\end{thebibliography}
\vspace{12pt}
\color{red}
IEEE conference templates contain guidance text for composing and formatting conference papers. Please ensure that all template text is removed from your conference paper prior to submission to the conference. Failure to remove the template text from your paper may result in your paper not being published.

\end{document}
