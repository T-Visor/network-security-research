\documentclass[11pt,conference]{IEEEtran}
\IEEEoverridecommandlockouts
% The preceding line is only needed to identify funding in the first footnote. If that is unneeded, please comment it out.
\usepackage{cite}
\usepackage{amsmath,amssymb,amsfonts}
\usepackage{graphicx}
\usepackage{textcomp}
\usepackage{xcolor}
\def\BibTeX{{\rm B\kern-.05em{\sc i\kern-.025em b}\kern-.08em
    T\kern-.1667em\lower.7ex\hbox{E}\kern-.125emX}}
\begin{document}

\title{Machine Learning Security Survey \\
}

\author{\IEEEauthorblockN{Turhan Kimbrough}
\IEEEauthorblockA{\textit{Department of Computer Science} \\
\textit{Towson University}\\
Towson, Maryland\\
tkimbr1@students.towson.edu}
}

\maketitle

\begin{abstract}
Machine learning has experienced a significant growth in usage 
over the past few decades. Due to its data-centric approach in modeling, 
machine learning has seen use in a variety of subfields in computer science. In
particular, researchers have been interested in incorporating machine learning
into the domain of cybersecurity, utilizing it from the perspective of an
adversary or ally. Researchers have also been concerned with the security state
of current machine learning models. This survey paper provides a comprehensive overview of
the state of machine learning, its application in various aspects of
cybersecurity, securing machine learning systems, and future research directions being explored.
\end{abstract}

\begin{IEEEkeywords}
Neural networks, security, classification
\end{IEEEkeywords}

\section{Introduction}
With the increasing widespread adoption of machine learning technology, its
usage is being observed in a variety of different fields. Some notable examples
include image recognition, voice assistant technologies, email spam filters, and search
engines. Much of its recent popularity can be attributed to the availability of
frameworks such as tensorflow, allowing people of almost any background to
quickly draft a machine learning application.
\newline

However, one growing concern tied to the ubiquity of machine learning is its 
accessibility to adversaries. Based on the assessment of current and prior
research, there are a number of vulnerabilities in current machine learning
models which can be exploited with little knowledge of a system's domain. In
addition, attackers have been able to leverage machine learning technology to
assist the deployment of cyberattacks. Enterprise machine learning applications
may often contain large datasets of important information, becoming a potential 
candidate of a targeted attack. This paper will survey the domain of
cybersecurity with regard to machine learning. This topic will explore the
fundamentals of machine learning, current vulnerabilities in machine learning
systems, machine learning technology from the perspective of adversaries, and
future research directions.

\section{Overview}
This section will introduce the fundamentals of machine learning. Additionally,
there will be a discussion about the key terms used by members of the community.

\subsection{Machine Learning Basics}
Traditionally, when software developers are tasked with solving a problem, they
use a combination of rules and logic to find a solution. The basic
routine consists of finding appropriate input values,
creating the logic and rules to process the input, and producing the appropriate output.
The traditional approach to software development allows for fine-tuned control
of program behavior to achieve the solution. However, this approach does not scale
with the complexity of additional rules and/or possible solutions. An
example is image classification, where the logic needed to compare images
is complex. This becomes a bigger concern when new classifications need to be
derived with new image data. Machine learning flips the traditional programming
paradigm on its head, by taking a series of solutions as input, and
letting the machine develop the rules by detecting patterns in the solutions.
The result is a self-propagating mathematical model, capable of making
decisions on newly supplied data. This approach relies on large sets of
well-defined data, and has the flexibility for being used in many different applications.

Briefly mentioned earlier, a common use case for machine learning is
\emph{classification}, where a dataset is categorized into different groups
based on one or more \emph{features}. A \emph{feature} is defined as some
measurable property or characteristic being observed in a dataset. There are several types of
classifications, including \emph{binary classification}, \emph{multi-class
classification}, and \emph{multi-label classification}. \emph{Binary
classification} categorizes data based on whether a feature is present or not,
resulting in an outcome of true or false. \emph{Multi-class classification}
categorizes data into different groups, where each data instance is assigned
according to its feature. \emph{Multi-label classification} categorizes data
into different groups, where each data instance is assigned according to its
expression of on or more features.

\emph{Training} is the process of teaching a machine learning model to detect patterns
in datasets. There are two types of training mechanisms, \emph{supervised
training} and \emph{unsupervised training}. \emph{Supervised training} requires
each data instance to have one or more labels, defining which category or
feature it expresses. In contrast, \emph{unsupervised training} omits the need
for labels, and the machine learning model will categorize datasets on its own.
Typically, after the training phase, a machine learning model will go through
the process of \emph{validation}. \emph{Validation} typically consists of
classifying a separate dataset to guarantee the accuracy of a model and
preventing a phenomenon called \emph{over-fitting}, where a model will only
'memorize' characteristics or patterns of training data.

\section{Vulnerabilities in Machine Learning Models}
This section will discuss several vulnerabilities which are currently present
in machine learning models.

\subsection{Data poisoning}
Since machine learning models are constructed based on the data fed to them,
the largest attack surface would be the training data itself. For an attacker
to manipulate a machine learning model, they would only need access to its
input mechanism. Data Poisoning is the act of manipulating, removing, or adding
data during the training phase of machine learning. This type of attack is
known as a \emph{black box attack}, where an attacker does not need to know the
implementation of a system to attack it.

One popular instance of a data poisoning attack is the adversarial
example. This requires the attacker to have some knowledge of the
training data, such as its dimensions and data type. The attacker would then
manipulate data instances in a way where the machine learning model would
be fooled, but appears normal to a human observer. The reason for manipulating
data instances in this way is two-fold. First, machine learning models are
often constrained to data fitting a specific dimension, shape, size, or length
of characters. This is typically done to prevent incompatible data from
entering the model during training. Second, there is often one or more people
who are observing the model with testing or validation datasets. An attacker
would want to minimize any evidence of the data being tampered with.

Often, the goal of this
attack is to make a machine learning model incorrectly classify data. This may
allow an attacker to sneak malicious data passed a system.

\subsection{}

\subsection{Equations}
Number equations consecutively. To make your 
equations more compact, you may use the solidus (~/~), the exp function, or 
appropriate exponents. Italicize Roman symbols for quantities and variables, 
but not Greek symbols. Use a long dash rather than a hyphen for a minus 
sign. Punctuate equations with commas or periods when they are part of a 
sentence, as in:
\begin{equation}
a+b=\gamma\label{eq}
\end{equation}

Be sure that the 
symbols in your equation have been defined before or immediately following 
the equation. Use ``\eqref{eq}'', not ``Eq.~\eqref{eq}'' or ``equation \eqref{eq}'', except at 
the beginning of a sentence: ``Equation \eqref{eq} is . . .''

\subsection{\LaTeX-Specific Advice}

Please use ``soft'' (e.g., \verb|\eqref{Eq}|) cross references instead
of ``hard'' references (e.g., \verb|(1)|). That will make it possible
to combine sections, add equations, or change the order of figures or
citations without having to go through the file line by line.

Please don't use the \verb|{eqnarray}| equation environment. Use
\verb|{align}| or \verb|{IEEEeqnarray}| instead. The \verb|{eqnarray}|
environment leaves unsightly spaces around relation symbols.

Please note that the \verb|{subequations}| environment in {\LaTeX}
will increment the main equation counter even when there are no
equation numbers displayed. If you forget that, you might write an
article in which the equation numbers skip from (17) to (20), causing
the copy editors to wonder if you've discovered a new method of
counting.

{\BibTeX} does not work by magic. It doesn't get the bibliographic
data from thin air but from .bib files. If you use {\BibTeX} to produce a
bibliography you must send the .bib files. 

{\LaTeX} can't read your mind. If you assign the same label to a
subsubsection and a table, you might find that Table I has been cross
referenced as Table IV-B3. 

{\LaTeX} does not have precognitive abilities. If you put a
\verb|\label| command before the command that updates the counter it's
supposed to be using, the label will pick up the last counter to be
cross referenced instead. In particular, a \verb|\label| command
should not go before the caption of a figure or a table.

Do not use \verb|\nonumber| inside the \verb|{array}| environment. It
will not stop equation numbers inside \verb|{array}| (there won't be
any anyway) and it might stop a wanted equation number in the
surrounding equation.

\subsection{Some Common Mistakes}\label{SCM}
\begin{itemize}
\item The word ``data'' is plural, not singular.
\item The subscript for the permeability of vacuum $\mu_{0}$, and other common scientific constants, is zero with subscript formatting, not a lowercase letter ``o''.
\item In American English, commas, semicolons, periods, question and exclamation marks are located within quotation marks only when a complete thought or name is cited, such as a title or full quotation. When quotation marks are used, instead of a bold or italic typeface, to highlight a word or phrase, punctuation should appear outside of the quotation marks. A parenthetical phrase or statement at the end of a sentence is punctuated outside of the closing parenthesis (like this). (A parenthetical sentence is punctuated within the parentheses.)
\item A graph within a graph is an ``inset'', not an ``insert''. The word alternatively is preferred to the word ``alternately'' (unless you really mean something that alternates).
\item Do not use the word ``essentially'' to mean ``approximately'' or ``effectively''.
\item In your paper title, if the words ``that uses'' can accurately replace the word ``using'', capitalize the ``u''; if not, keep using lower-cased.
\item Be aware of the different meanings of the homophones ``affect'' and ``effect'', ``complement'' and ``compliment'', ``discreet'' and ``discrete'', ``principal'' and ``principle''.
\item Do not confuse ``imply'' and ``infer''.
\item The prefix ``non'' is not a word; it should be joined to the word it modifies, usually without a hyphen.
\item There is no period after the ``et'' in the Latin abbreviation ``et al.''.
\item The abbreviation ``i.e.'' means ``that is'', and the abbreviation ``e.g.'' means ``for example''.
\end{itemize}
An excellent style manual for science writers is \cite{b7}.

\subsection{Authors and Affiliations}
\textbf{The class file is designed for, but not limited to, six authors.} A 
minimum of one author is required for all conference articles. Author names 
should be listed starting from left to right and then moving down to the 
next line. This is the author sequence that will be used in future citations 
and by indexing services. Names should not be listed in columns nor group by 
affiliation. Please keep your affiliations as succinct as possible (for 
example, do not differentiate among departments of the same organization).

\subsection{Identify the Headings}
Headings, or heads, are organizational devices that guide the reader through 
your paper. There are two types: component heads and text heads.

Component heads identify the different components of your paper and are not 
topically subordinate to each other. Examples include Acknowledgments and 
References and, for these, the correct style to use is ``Heading 5''. Use 
``figure caption'' for your Figure captions, and ``table head'' for your 
table title. Run-in heads, such as ``Abstract'', will require you to apply a 
style (in this case, italic) in addition to the style provided by the drop 
down menu to differentiate the head from the text.

Text heads organize the topics on a relational, hierarchical basis. For 
example, the paper title is the primary text head because all subsequent 
material relates and elaborates on this one topic. If there are two or more 
sub-topics, the next level head (uppercase Roman numerals) should be used 
and, conversely, if there are not at least two sub-topics, then no subheads 
should be introduced.

\subsection{Figures and Tables}
\paragraph{Positioning Figures and Tables} Place figures and tables at the top and 
bottom of columns. Avoid placing them in the middle of columns. Large 
figures and tables may span across both columns. Figure captions should be 
below the figures; table heads should appear above the tables. Insert 
figures and tables after they are cited in the text. Use the abbreviation 
``Fig.~\ref{fig}'', even at the beginning of a sentence.

\begin{table}[htbp]
\caption{Table Type Styles}
\begin{center}
\begin{tabular}{|c|c|c|c|}
\hline
\textbf{Table}&\multicolumn{3}{|c|}{\textbf{Table Column Head}} \\
\cline{2-4} 
\textbf{Head} & \textbf{\textit{Table column subhead}}& \textbf{\textit{Subhead}}& \textbf{\textit{Subhead}} \\
\hline
copy& More table copy$^{\mathrm{a}}$& &  \\
\hline
\multicolumn{4}{l}{$^{\mathrm{a}}$Sample of a Table footnote.}
\end{tabular}
\label{tab1}
\end{center}
\end{table}

\begin{figure}[htbp]
\centerline{\includegraphics{fig1.png}}
\caption{Example of a figure caption.}
\label{fig}
\end{figure}

Figure Labels: Use 8 point Times New Roman for Figure labels. Use words 
rather than symbols or abbreviations when writing Figure axis labels to 
avoid confusing the reader. As an example, write the quantity 
``Magnetization'', or ``Magnetization, M'', not just ``M''. If including 
units in the label, present them within parentheses. Do not label axes only 
with units. In the example, write ``Magnetization (A/m)'' or ``Magnetization 
\{A[m(1)]\}'', not just ``A/m''. Do not label axes with a ratio of 
quantities and units. For example, write ``Temperature (K)'', not 
``Temperature/K''.

\section*{Acknowledgment}

The preferred spelling of the word ``acknowledgment'' in America is without 
an ``e'' after the ``g''. Avoid the stilted expression ``one of us (R. B. 
G.) thanks $\ldots$''. Instead, try ``R. B. G. thanks$\ldots$''. Put sponsor 
acknowledgments in the unnumbered footnote on the first page.

\section*{References}

Please number citations consecutively within brackets \cite{b1}. The 
sentence punctuation follows the bracket \cite{b2}. Refer simply to the reference 
number, as in \cite{b3}---do not use ``Ref. \cite{b3}'' or ``reference \cite{b3}'' except at 
the beginning of a sentence: ``Reference \cite{b3} was the first $\ldots$''

Number footnotes separately in superscripts. Place the actual footnote at 
the bottom of the column in which it was cited. Do not put footnotes in the 
abstract or reference list. Use letters for table footnotes.

Unless there are six authors or more give all authors' names; do not use 
``et al.''. Papers that have not been published, even if they have been 
submitted for publication, should be cited as ``unpublished'' \cite{b4}. Papers 
that have been accepted for publication should be cited as ``in press'' \cite{b5}. 
Capitalize only the first word in a paper title, except for proper nouns and 
element symbols.

For papers published in translation journals, please give the English 
citation first, followed by the original foreign-language citation \cite{b6}.

\begin{thebibliography}{00}
\bibitem{b1} G. Eason, B. Noble, and I. N. Sneddon, ``On certain integrals of Lipschitz-Hankel type involving products of Bessel functions,'' Phil. Trans. Roy. Soc. London, vol. A247, pp. 529--551, April 1955.
\bibitem{b2} J. Clerk Maxwell, A Treatise on Electricity and Magnetism, 3rd ed., vol. 2. Oxford: Clarendon, 1892, pp.68--73.
\bibitem{b3} I. S. Jacobs and C. P. Bean, ``Fine particles, thin films and exchange anisotropy,'' in Magnetism, vol. III, G. T. Rado and H. Suhl, Eds. New York: Academic, 1963, pp. 271--350.
\bibitem{b4} K. Elissa, ``Title of paper if known,'' unpublished.
\bibitem{b5} R. Nicole, ``Title of paper with only first word capitalized,'' J. Name Stand. Abbrev., in press.
\bibitem{b6} Y. Yorozu, M. Hirano, K. Oka, and Y. Tagawa, ``Electron spectroscopy studies on magneto-optical media and plastic substrate interface,'' IEEE Transl. J. Magn. Japan, vol. 2, pp. 740--741, August 1987 [Digests 9th Annual Conf. Magnetics Japan, p. 301, 1982].
\bibitem{b7} M. Young, The Technical Writer's Handbook. Mill Valley, CA: University Science, 1989.
\end{thebibliography}
\vspace{12pt}
\color{red}
IEEE conference templates contain guidance text for composing and formatting conference papers. Please ensure that all template text is removed from your conference paper prior to submission to the conference. Failure to remove the template text from your paper may result in your paper not being published.

\end{document}
