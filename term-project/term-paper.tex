\documentclass[11pt,conference]{IEEEtran}
\IEEEoverridecommandlockouts
% The preceding line is only needed to identify funding in the first footnote. If that is unneeded, please comment it out.
\usepackage{cite}
\usepackage{amsmath,amssymb,amsfonts}
\usepackage{graphicx}
\usepackage{textcomp}
\usepackage{xcolor}

\usepackage[english]{babel}
\usepackage[utf8]{inputenc}
\usepackage{algorithm}
\usepackage[algo2e]{algorithm2e} 
\usepackage[noend]{algpseudocode}


\def\BibTeX{{\rm B\kern-.05em{\sc i\kern-.025em b}\kern-.08em
\kern-.1667em\lower.7ex\hbox{E}\kern-.125emX}}
\begin{document}

\title{Using Machine Learning To Solve Text-based CAPTCHAs\\
}

\author{\IEEEauthorblockN{Turhan Kimbrough}
\IEEEauthorblockA{\textit{Department of Computer Science} \\
\textit{Towson University}\\
Towson, Maryland\\
tkimbr1@students.towson.edu}
}

\maketitle

\begin{abstract}
	CAPTCHA is an acronym for the "Completely Automated Public Turing test
	to tell Computers and Humans Apart". It is a mechanism which is used to
	distinguish real human users from bots. CAPTCHAs come in a variety of forms,
	including the deciphering of obfuscated text, transcribing of audio messages,
	tracking mouse movement, and more. This research will focus on automating the
	process of deciphering text-based CAPTCHAs using machine learning
	techniques. Specifically, supervised learning and convolutional neural
	networks are used to develop
	a model which is capable of over 99\% accuracy for certain datasets. 
	The goal of this research is to demonstrate the weaknesses associated with text-based
	CAPTCHA mechanisms, especially with the prevalence of machine learning
	tools.
\end{abstract}

\begin{IEEEkeywords}
	machine learning, neural networks, supervised training, CAPTCHA
\end{IEEEkeywords}

\section{Introduction}
CAPTCHA is an acronym for the "\textbf{C}ompletely \textbf{A}utomated
\textbf{P}ublic \textbf{T}uring test to tell
\textbf{C}omputers and \textbf{H}umans \textbf{A}part", which is a
challenge-response test used in computing services to verify that the user is a
human. The premise of a CAPTCHA is to provide a test which is relatively easy
for a human to solve, but difficult for bots. This is one of many 
mechanisms used to combat against the growing usage of malicious software
automation. Due to the ubiquity of automation software, cybercriminals have
been able to easily create bots to perform malicious acts. These acts include
denial-of-service attacks, autonomous social media communication agents,
scalping scarce merchandise, and more.

Due to the wide availability of CAPTCHA-generating software, they have become a popular
mechanism to integrate into websites. In particular, text-based CAPTCHAs are
often available as a low-cost and simple solution. Text-based CAPTCHAs
typically consist of alphanumeric characters in an image, which has been
manipulated to prevent it from being easily parsed by a machine. Users are then
challenged to decipher the text in the CAPTCHA, and if the answer is correct,
they can continue using the service. CAPTCHAs are typically available from
content management systems (such as WordPress) or can be integrated into a
website via API.

\begin{figure}[htbp]
	\centerline{\includegraphics[scale=0.7]{images/alphanumeric-captcha.png}}
	\caption{Example of a text-based CAPTCHA.}
	\label{figure}
\end{figure}

While this mechanism can mitigate the majority of software bots, it is not
effective against bots utilizing machine learning technology. This research
paper demonstrates that a machine learning agent is capable of solving CAPTCHAs
with the use of open-source tools, supervised training, and convolutional
neural networks. The goal of this research is show how adversaries can use
readily available technologies to exploit text-based CAPTCHA mechanisms.
This paper will cover background/related work, the methodology  used for solving CAPTCHAs,
challenges which were present, key contributions, results, and a section
covering future work.

\section{Background/Related Work}
In this section, there will be a brief review of similar work which has been
done on using machine learning to solve CAPTCHA tests.

\subsection{Solving reCAPTCHAs With Reinforcement Learning}
Researchers at [] have demonstrated the ability to solve mouse-based reCAPTCHAs using
reinforcement learning. Google's reCAPTCHA mechanism is more difficult to solve
compared to traditional CAPTCHAs due to its usage of mouse-tracking to
determine if the user is a human. While the exact algorithm is unknown due to
the closed-source nature of the reCAPTCHA technology, the researchers uses a
black-box approach to solve reCAPTCHAs.

The approach models mouse movements as transitions on a 2-dimensional grid
of pixels. The \emph{Markov Decision Process} is used to generate a series of
movements (up, down, left, right), which is combination of random and controlled
outcomes. The mouse-control agent is then trained through reinforcement
learning to generate a series of movements which mimic the behavior of humans.
This methodology was able to achieve a success rate of 97.4\% on a 100x100
grid and 96.7\% on a 1000x1000 display.

\begin{figure}[htbp]
	\centerline{\includegraphics[scale=0.5]{images/grid-world.png}}
	\caption{Grid world model.}
	\label{figure}
\end{figure}

\subsection{Generic Solving of Text-based CAPTCHAs}
Researchers at [] have provided a basic framework on solving CAPTCHAs using
segmentation and character recognition techniques. More specifically, their
technique is able to detect individual characters in CAPTCHAs with occluding
lines. Traditional approaches to CAPTCHA-solving typically use two separate
algorithms for segmentation and character recognition. The researchers have
instead, developed an algorithm which combines the steps of segmentation and character
recognition.

The researchers approached this solution by studying previous schemes which
were used to analyze characters in CAPTCHAs. Many of them were unable to
segment CAPTCHAs which used \emph{negative kerning}, a technique
where negative space is used between characters to ensure occlusion 
by their neighbors. The algorithm developed by the researchers use machine
learning to perform 3 steps: \emph{cut-point detection}, \emph{slicing}, and
\emph{scoring}.
Cut-point detection analyzes all combinations of character partitioning. The
slicer will then segment each character using every partitioning combination,
placing them on a graph afterwards. Finally, the scorer assigns a weight for
each partitioning combination and determines which characters are most likely
present in the CAPTCHA.

\begin{figure}[htbp]
	\centerline{\includegraphics[scale=0.5]{images/negative-kerning.png}}
	\caption{Example of negative kerning.}
	\label{figure}
\end{figure}

\subsection{Immutable Adversarial Examples for CAPTCHA Generation}
While the two works above demonstrate the exploitation of CAPTCHAs, the work
presented here will demonstrate an approach for securing CAPTCHAs. Researchers
at [] have demonstrated the ability to produce CAPTCHAs which are resistant to
noise-removal attempts. Often, CAPTCHA solving techniques using neural networks
apply filters to a CAPTCHA image to reduce noise and exaggerate feature
sets. Generating CAPTCHAs with immutable noise would pose a significant
challenge to many of the machine learning assisted CAPTCHA solvers.

The researchers were able to create immutable adversarial CAPTCHAs by using a
modification of the \emph{fast gradient sign} (FGS) method. The FGS method
works by slightly altering the gradient values of certain image pixels to
maximize \emph{loss} during the training process of a neural network. The
modified FGS method takes target label and confidence level as inputs in
addition to the original image. This allows for noise generation to work
towards a specific goal, distributing the noise in such a way that is
difficult to filter.

\begin{figure}[htbp]
	\centerline{\includegraphics[scale=0.35]{images/fgsm_technique.png}}
	\caption{FGS method fooling a machine learning model.}
	\label{figure}
\end{figure}

\section{Methodology}

In this section, a proof-of-concept CAPTCHA-solving model is constructed using
open-source tools and machine learning principles. The first two subsections
will give background information on the tools/principles being applied. The
rest of the subsections will walk through the procedure for creating the machine
learning model.

\subsection{Open-source Tool Selection}
\emph{Python 3} will be the programming language of choice, due to its
easy-to-use syntax, portability, and wide range of modules. To complement
Python 3, the \emph{Python Image Library (PIL)} will be used to generate
CAPTCHA images, and \emph{TensorFlow} will be the core library for building the
machine learning model. Lastly, the code will be written for the \emph{Jupyter
Notebook} environment, a popular open-source web application for creating/sharing
documents in the scientific community.

\subsection{Applied Machine Learning Principles}
Two core machine learning principles will be applied when creating the
CAPTCHA-solving model, \emph{convolutional neural networks (CNN's)} and
\emph{supervised learning}. These two principles are commonly used for
image-processing, a perfect use-case for CAPTCHA-solving.

CNN's are a subset of \emph{artificial neural networks (ANN's)}, a family of algorithms
which mimic the structure of biological neural networks found in animal brains.
ANN's consist of an input layer for data, one or more hidden layers for
data-processing, and an output layer for decision-making. CNN's use
a combination of \emph{convolutional layers} and \emph{pooling layers} to
represent the hidden layers in its structure. Convolutional layers are used to
create feature maps, a technique used to extract characteristics from image
data. A pooling layer is placed after each convolutional layer to reduce the
size of each feature map, lowering the computation power required for
further processing.

Supervised learning is a technique to train a machine learning model with
the use of labelled data. The labels in the dataset define a category or
feature for each data instance.

\begin{figure}[htbp]
	\centerline{\includegraphics[scale=0.25]{images/cnn-example.png}}
	\caption{FGS Example structure of convolutional neural network.}
	\label{figure}
\end{figure}


\subsection{Creating the Training Data}
Creating the training dataset consists of two parts; generating a large series
of CAPTCHA images and creating labels for each CAPTCHA image. In order to
satisfy these two requirements, PIL will be used to generate CAPTCHA images
with their labels. 

In a script using PIL, a total of 10,000 images will be generated to cover all
possible variations of 4-digit numbers. Each image will use a dimension of 
100x100 pixels and use a consistent font \emph{(DejaVu Sans)}. To obfuscate the
text, each image will contain random colored dots and lines. A different color
will also be used to print the text in each image. Afterwards, the image will be saved
as a PNG image file, with the 4-digit string included in the file name. All
CAPTCHA images will be saved in the same directory for processing later on.

\begin{algorithm}
	\caption{Generating labelled CAPTCHAs}
		\State $count \leftarrow 0 $
		\State
		\While{$count < 10,000$} { 
		\State $number \leftarrow \Call{GetFourDigitString}{count} $
		\State $font \leftarrow \Call{GetSystemFont}{\null} $
		\State $captcha \leftarrow \Call{CreateImageWithText}{font, number} $
        \State \Call{Resize}{$captcha, 100, 100} $
        \State \Call {ColorText}{$captcha} $
		\State \Call{DrawColoredLines}{$captcha} $
		\State \Call{DrawColoredDots}{$captcha} $
		\State $captcha\_name \leftarrow number + "\_image.png" $
		\State \Call{Save}{$captcha, captcha\_name} $
		\State $count \leftarrow count + 1 $
		\EndWhile}
\end{algorithm}

\begin{figure}[htbp]
	\centerline{\includegraphics[scale=1.5]{images/0058_image.png}}
	\caption{0058\_image.png}
	\label{figure}
\end{figure}

The rest of the model-building procedure will take place in the \emph{Jupyter
Notebook}
file. \emph{TensorFlow} will need to be imported along with a helper library,
\emph{pandas}. The pandas library contains many data structures which are
commonly used with TensorFlow. Since the CAPTCHA images generated in
the PIL script will need to organized for TensorFlow to work with, the pandas
\emph{DataFrame} structure will be used. The DataFrame is a tabular data
structure with rows denoting iterations and columns representing data
attributes.

All CAPTCHA images will be imported into the Jupyter Notebook file by obtaining
the file path to the directory that they were stored. A function will then
be called repeatedly to parse each file path and obtain the 4-digit CAPTCHA string associated with each
CAPTCHA image. The pandas DataFrame will then be used to store a pair of values
for each row, the CAPTCHA image path and the 4-digit CAPTCHA string. This
associates the label with its respective data.


\begin{algorithm}[]
  \SetKwInOut{Input}{Input}
  \SetKwInOut{Output}{Output}
  \SetKwProg{try}{try}{:}{}
  \SetKwProg{catch}{catch}{:}{end}
  \Input{String $file\_path}
  \Output{String $label}

  \try{}{
    $file\_name \leftarrow \Call{SplitFileName}{file\_path} $

	\Call{SplitExtension}{$file\_name, ".png"} $

	$label \leftarrow \Call{Split}{file\_name, "\_"} $

  	\Return $label $
  }
  \catch{InvalidPathException}{
    \Return $UNSUCCESSFUL $
  }
  \caption{Get label for CAPTCHA image}
  \label{alg:exep}
\end{algorithm}

\subsection{Defining the Neural Network Structure}
In addition to TensorFlow, the library \emph{Keras} will be used to define the
layers of the neural network. The purpose of using this neural network is to extract the features
(the sequence of digits) which are present in each CAPTCHA image.

The first layer of the neural network is the \emph{input layer}, which will be
the entry point for image data. Here, the input layer will specify each CAPTCHA image
to have a height and width of 100 pixels, and detect 3 color channels of
(red, green, and blue). This will reject anomalies presented to the machine
learning model and enforce uniformity.

The \emph{hidden layers} are responsible for extracting the features in each
image and recognizing patterns in the dataset.

\subsection{Training the Machine Learning Model}

\section{Results}

\section{Challenges (mention caveats here)}

\section{Key Contributions}



\section{Future Work}


\begin{thebibliography}{00}
	\bibitem{b1} G. Eason, B. Noble, and I. N. Sneddon, ``On certain integrals of Lipschitz-Hankel type involving products of Bessel functions,'' Phil. Trans. Roy. Soc. London, vol. A247, pp. 529--551, April 1955.
	\bibitem{b2} J. Clerk Maxwell, A Treatise on Electricity and Magnetism, 3rd ed., vol. 2. Oxford: Clarendon, 1892, pp.68--73.
	\bibitem{b3} I. S. Jacobs and C. P. Bean, ``Fine particles, thin films and exchange anisotropy,'' in Magnetism, vol. III, G. T. Rado and H. Suhl, Eds. New York: Academic, 1963, pp. 271--350.
	\bibitem{b4} K. Elissa, ``Title of paper if known,'' unpublished.
	\bibitem{b5} R. Nicole, ``Title of paper with only first word capitalized,'' J. Name Stand. Abbrev., in press.
	\bibitem{b6} Y. Yorozu, M. Hirano, K. Oka, and Y. Tagawa, ``Electron spectroscopy studies on magneto-optical media and plastic substrate interface,'' IEEE Transl. J. Magn. Japan, vol. 2, pp. 740--741, August 1987 [Digests 9th Annual Conf. Magnetics Japan, p. 301, 1982].
	\bibitem{b7} M. Young, The Technical Writer's Handbook. Mill Valley, CA: University Science, 1989.
\end{thebibliography}
\vspace{12pt}
\color{red}
IEEE conference templates contain guidance text for composing and formatting conference papers. Please ensure that all template text is removed from your conference paper prior to submission to the conference. Failure to remove the template text from your paper may result in your paper not being published.

\end{document}
